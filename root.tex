%%%%%%%%%%%%%%%%%%%%%%%%%%%%%%%%%%%%%%%%%%%%%%%%%%%%%%%%%%%%%%%%%%%%%%%%%%%%%%%%
%2345678901234567890123456789012345678901234567890123456789012345678901234567890
%        1         2         3         4         5         6         7         8

\documentclass[letterpaper, 10 pt, conference]{ieeeconf}  % Comment this line out if you need a4paper
\usepackage{graphicx}

%\documentclass[a4paper, 10pt, conference]{ieeeconf}      % Use this line for a4 paper

\IEEEoverridecommandlockouts                              % This command is only needed if 
                                                          % you want to use the \thanks command

\overrideIEEEmargins                                      % Needed to meet printer requirements.

%In case you encounter the following error:
%Error 1010 The PDF file may be corrupt (unable to open PDF file) OR
%Error 1000 An error occurred while parsing a contents stream. Unable to analyze the PDF file.
%This is a known problem with pdfLaTeX conversion filter. The file cannot be opened with acrobat reader
%Please use one of the alternatives below to circumvent this error by uncommenting one or the other
%\pdfobjcompresslevel=0
%\pdfminorversion=4

% See the \addtolength command later in the file to balance the column lengths
% on the last page of the document

% The following packages can be found on http:\\www.ctan.org
%\usepackage{graphics} % for pdf, bitmapped graphics files
%\usepackage{epsfig} % for postscript graphics files
%\usepackage{mathptmx} % assumes new font selection scheme installed
%\usepackage{times} % assumes new font selection scheme installed
%\usepackage{amsmath} % assumes amsmath package installed
%\usepackage{amssymb}  % assumes amsmath package installed

\title{\LARGE \bf
CSE 598 Project Proposal: \\
Imitation Learning with Baxter Robot using Hi-Fives
}

\author{Michael Drolet, Frankie Liu, Evan Lam}


\begin{document}

\maketitle
\thispagestyle{empty}
\pagestyle{empty}


%%%%%%%%%%%%%%%%%%%%%%%%%%%%%%%%%%%%%%%%%%%%%%%%%%%%%%%%%%%%%%%%%%%%%%%%%%%%%%%%
\begin{abstract}

The goal of this project is to successfully learn hi-fives through human-robot interaction. We will be using an Imitation Learning approach that incorporates Bayesian Interaction Primitives (reference to Joe). Through this experiment, we aim to develop a responsive interaction with the robot.

\end{abstract}


%%%%%%%%%%%%%%%%%%%%%%%%%%%%%%%%%%%%%%%%%%%%%%%%%%%%%%%%%%%%%%%%%%%%%%%%%%%%%%%%
\section{INTRODUCTION}
Teaching robots how to learn new tasks and interact with humans can be done using a variety of methods. Imitation learning, which uses a human expert to guide the interaction, is one popular approach that we seek to use for this experiment.


\section{Related Work}

\subsection{Bayesian Interaction Primitives}

Joe Campbell, int prim

\subsection{Reinforcement Learning}

Find some RL paper


\section{Problem Statement}
Create a robot that can hi-five in vrep and in person.

\subsection{Simulation} 
We will do simulations in VREP.


\subsection{In person}

We will do in person experiments with the actual robot.

\section{Experiments}
Discuss experiments

\subsection{Data Generation}
Optitrack, ROS, etc.

\subsection{Biomechanics}
Discuss biomechanics aspect

\subsection{Domains}
Discuss the domain of our project

\subsection{TBD section}
TBD if we need more space.

\section{Parameter Tuning}
We had to tune some parameters.

\section{Discussion and Analysis}
Not applicable yet.


\section{CONCLUSIONS}
Not sure about this either.

\addtolength{\textheight}{-12cm}   % This command serves to balance the column lengths
                                  % on the last page of the document manually. It shortens
                                  % the textheight of the last page by a suitable amount.
                                  % This command does not take effect until the next page
                                  % so it should come on the page before the last. Make
                                  % sure that you do not shorten the textheight too much.

%%%%%%%%%%%%%%%%%%%%%%%%%%%%%%%%%%%%%%%%%%%%%%%%%%%%%%%%%%%%%%%%%%%%%%%%%%%%%%%%



%%%%%%%%%%%%%%%%%%%%%%%%%%%%%%%%%%%%%%%%%%%%%%%%%%%%%%%%%%%%%%%%%%%%%%%%%%%%%%%%



%%%%%%%%%%%%%%%%%%%%%%%%%%%%%%%%%%%%%%%%%%%%%%%%%%%%%%%%%%%%%%%%%%%%%%%%%%%%%%%%

\begin{thebibliography}{99}
\bibitem{c1} Chris Paxton, Vasumathi Raman, Gregory D Hager, and Marin Kobilarov. Combining neural networks and tree search for task and motion planning in challenging environments. ArXiv e-prints, 2017.
\end{thebibliography}




\end{document}
