%%%%%%%%%%%%%%%%%%%%%%%%%%%%%%%%%%%%%%%%%%%%%%%%%%%%%%%%%%%%%%%%%%%%%%%%%%%%%%%%
%2345678901234567890123456789012345678901234567890123456789012345678901234567890
%        1         2         3         4         5         6         7         8

\documentclass[letterpaper, 10 pt, conference]{ieeeconf}  % Comment this line out if you need a4paper
\usepackage{graphicx}

%\documentclass[a4paper, 10pt, conference]{ieeeconf}      % Use this line for a4 paper

\IEEEoverridecommandlockouts                              % This command is only needed if 
                                                          % you want to use the \thanks command

\overrideIEEEmargins                                      % Needed to meet printer requirements.

%In case you encounter the following error:
%Error 1010 The PDF file may be corrupt (unable to open PDF file) OR
%Error 1000 An error occurred while parsing a contents stream. Unable to analyze the PDF file.
%This is a known problem with pdfLaTeX conversion filter. The file cannot be opened with acrobat reader
%Please use one of the alternatives below to circumvent this error by uncommenting one or the other
%\pdfobjcompresslevel=0
%\pdfminorversion=4

% See the \addtolength command later in the file to balance the column lengths
% on the last page of the document

% The following packages can be found on http:\\www.ctan.org
%\usepackage{graphics} % for pdf, bitmapped graphics files
%\usepackage{epsfig} % for postscript graphics files
%\usepackage{mathptmx} % assumes new font selection scheme installed
%\usepackage{times} % assumes new font selection scheme installed
%\usepackage{amsmath} % assumes amsmath package installed
%\usepackage{amssymb}  % assumes amsmath package installed

\title{\LARGE \bf
CSE 598 Project Proposal: \\
Imitation Learning with Baxter Robot using Hi-Fives
}

\author{Michael Drolet, Frankie Liu, Evan Lam}


\begin{document}

\maketitle
\thispagestyle{empty}
\pagestyle{empty}


%%%%%%%%%%%%%%%%%%%%%%%%%%%%%%%%%%%%%%%%%%%%%%%%%%%%%%%%%%%%%%%%%%%%%%%%%%%%%%%%
\begin{abstract}
The goal of this project is to successfully learn hi-fives for human-robot interaction. We will be using an Imitation Learning approach by incorporating Bayesian Interaction Primitives \cite{c1}. Through expert-guided demonstrations, we train the robot to learn relationships between human and robot trajectories. We demonstrate that the robot is able to complete the interaction with a human and successfully issue a hi-five.
\end{abstract}


%%%%%%%%%%%%%%%%%%%%%%%%%%%%%%%%%%%%%%%%%%%%%%%%%%%%%%%%%%%%%%%%%%%%%%%%%%%%%%%%
\section{INTRODUCTION}
Teaching robots how to interact with humans is an interesting challenge in the field of Human Robot Interaction. The athletic and social components which make humans successful in daily interactions are often difficult to capture. Imitation Learning is a useful tool for these situations, where we are able to capture the ideal actions performed by a human and partner. However, teaching robots how to learn new tasks and interact with humans can be done using a variety of methods. Imitation learning, which uses a human expert to guide the interaction, is one popular approach that we seek to use for this experiment. Reinforcement Learning is another popular option for learning tasks that humans are experienced at. We hypothesize that a Reinforcement Learning approach over a continuous and high-dimensional state space will require a significant number of demonstrations in order for the robot to learn the hi-five interaction, compared to the Bayesian Interaction approach. As a result, for a simple interaction that is able to be generalized across different robots using limited samples, we found the Bayesian Interaction approach to be suitable for our project.
\newline
\indent
Throughout the semester in CSE598, we were introduced to a number of concepts in Imitation Learning. Dynamic Motor Primitives, which are a powerful tool for Imitation Learning, are compelling due to their biologically-inspired approach. We saw in the case of the frog example, how linear dynamical systems can be created to model the trajectory of an object. However, since many systems have nonlinear behaviour, there is a need to add a forcing function. The forcing function makes the dynamical system nonlinear, and as a drawback, things can quickly get unstable. We hypothesize that this may not be an issue with a more simplistic hi-five interaction, so the DMP approach is also attractive. Ultimately, the uncertainty of how much control theory and differential equations are needed to use DMP's led us to consider other options. On the other hand, we are confident that we could model the forcing by learning weights for the basis function decompositions of a trajectory, as this is what we are doing in our current approach.
\section{Related Work}

\subsection{Bayesian Interaction Primitives}
\indent The Bayesian Interaction Primitive (BIP) framework is a novel and powerful approach that is being used by the Interactive Robotics Lab at Arizona State University. This SLAM inspired algorithm is useful for encoding spatiotemporal information about the interaction into a state space. BIP approximates each dimension using a weighted linear combination of time-dependent basis functions. In this case, we are using Gaussian basis functions. Included in a state vector are the weights for every basis function, the phase, and phase velocity. As a result, BIP allows us to infer the current state, given previous states (demonstrations) and current observations. Formatting this probabilistically allows the inference to be done using Bayes Filters (in particular, a Kalman Filter) during testing to guide the robot. Since the robot's degrees of freedom are no longer being observed during testing, the BIP framework is based on obtaining a partial observation of the current state to generate the positions/variables of the whole state space. This approach is effective for learning on different types of robots, as there is no domain specific knowledge encoded into the algorithm.
\subsection{Reinforcement Learning}

Find some RL paper


\section{Problem Statement}
\indent Our Group has decided to use a hybrid approach- that is, using simulation and in-person examples- to present our demonstrations. This includes capturing data in real world through the Optitrack motion capture system.

\subsection{Simulation} 
We will do simulations in VREP.


\subsection{In person}

We will do in person experiments with the actual robot.

\section{Experiments}
Discuss experiments

\subsection{Data Generation}
Optitrack, ROS, etc.

\subsection{Biomechanics}
Discuss biomechanics aspect

\subsection{Domains}
Discuss the domain of our project

\subsection{TBD section}
TBD if we need more space.

\section{Parameter Tuning}
We had to tune some parameters.

\section{Discussion and Analysis}
Not applicable yet.


\section{CONCLUSIONS}
Not sure about this either.

\addtolength{\textheight}{-12cm}   % This command serves to balance the column lengths
                                  % on the last page of the document manually. It shortens
                                  % the textheight of the last page by a suitable amount.
                                  % This command does not take effect until the next page
                                  % so it should come on the page before the last. Make
                                  % sure that you do not shorten the textheight too much.

%%%%%%%%%%%%%%%%%%%%%%%%%%%%%%%%%%%%%%%%%%%%%%%%%%%%%%%%%%%%%%%%%%%%%%%%%%%%%%%%



%%%%%%%%%%%%%%%%%%%%%%%%%%%%%%%%%%%%%%%%%%%%%%%%%%%%%%%%%%%%%%%%%%%%%%%%%%%%%%%%



%%%%%%%%%%%%%%%%%%%%%%%%%%%%%%%%%%%%%%%%%%%%%%%%%%%%%%%%%%%%%%%%%%%%%%%%%%%%%%%%

\begin{thebibliography}{99}
\bibitem{c1} Chris Paxton, Vasumathi Raman, Gregory D Hager, and Marin Kobilarov. Combining neural networks and tree search for task and motion planning in challenging environments. ArXiv e-prints, 2017.
\end{thebibliography}




\end{document}
